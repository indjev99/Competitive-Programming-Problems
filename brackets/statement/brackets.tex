\documentclass[12pt]{article}
%---DOCUMENT MARGINS---
\usepackage{geometry} % Required for adjusting page dimensions and margins
\geometry{
	paper=a4paper, % Paper size, change to letterpaper for US letter size
	top=4cm, % Top margin
	bottom=\bottomMargin, % Bottom margin
	left=2cm, % Left margin
	right=2cm, % Right margin
	headheight=3cm, % Header height
	%footskip=1.5cm, % Space from the bottom margin to the baseline of the footer
	headsep=0.5cm, % Space from the top margin to the baseline of the header
	%showframe, % Uncomment to show how the type block is set on the page
}
\usepackage{cmap}

\usepackage[T2A]{fontenc}
\usepackage[bulgarian]{babel}
\usepackage{fontspec}
\setmainfont{Times New Roman}
\setsansfont{Times New Roman}
\setmonofont{Courier New}
\usepackage[math-style=TeX]{unicode-math}
\setmathfont{Latin Modern Math}

\usepackage[nobottomtitles*]{titlesec}
\titleformat
{\section} % command
{\normalfont\fontsize{14}{14}\sffamily\bfseries} % format
{} % label
{0pt} % sep
{} % before-code
\titlespacing{\section}{0pt}{0em}{0em}
\usepackage[dvipsnames]{xcolor}
\titleformat
{\subsection} % command
{\fontsize{14}{14}\itshape} % format
{} % label
{0pt} % sep
{} % before-code
[\vspace{-1em}{\color{LimeGreen}\rule{0.2\textwidth}{0.2em}}\vspace{-0.7em}] % after-code
\titlespacing{\subsection}{0pt}{0.5em}{0em}

\setlength{\parskip}{0.5em}
\setlength{\parindent}{24pt}
\sloppy

\usepackage{fancyhdr}
\pagestyle{fancy}
\usepackage{setspace}
\fancyhead[L]{
	\begin{minipage}{\textwidth}
		\includegraphics[width=2.8cm]{./logo.jpg}
	\end{minipage}
}
\fancyhead[C]{
	\begin{minipage}{\textwidth}
		\centering\large{\bf{\header}}
		\vspace{-0.35cm}
	\end{minipage}
}
\usepackage{emoji}
\usepackage{makecell}
\usepackage{tabularray}
\AtBeginEnvironment{table}{\vspace{-0.2cm}}
\AtEndEnvironment{table}{\vspace{-0.2cm}}
\usepackage{float}
\fancyhead[R]{
	\begin{tabular}{r@{\hspace{0.2cm}}l}
		\emoji{hourglass-not-done}: & \tl \\ 
		\emoji{floppy-disk}: & \ml \\ 
	\end{tabular}%
}
\renewcommand{\headrulewidth}{0cm}
\fancyheadoffset[L]{1cm}
\fancyheadoffset[R]{1cm}

\raggedbottom

\usepackage{amsmath}
\usepackage{stmaryrd}

\usepackage{graphicx}
\graphicspath{{./}}
\usepackage[export]{adjustbox}
\usepackage{wrapfig}
\makeatletter
\patchcmd\WF@putfigmaybe{\lower\intextsep}{}{}{\fail}
\AddToHook{env/wrapfigure/begin}{\setlength{\intextsep}{0pt}}
\makeatother
\usepackage[inkscapearea=page,inkscapepath=./svg-inkscape]{svg}
\svgpath{{./}}

\usepackage{placeins}
\usepackage{caption}
\captionsetup[table]{
	skip=0.25em,font=it,
	singlelinecheck=false,justification=justified,indention=-24pt,
	margin={24pt, 0pt}
}

\usepackage{enumitem}
\setlist{itemsep=-0.4em,leftmargin=\parindent,topsep=-\parskip}
\newcommand{\tabitem}{\indent~~\llap{\textbullet}~~}

\usepackage{hyperref}
\hypersetup{
	colorlinks=true,
	citecolor=blue,
	linkcolor=blue,
	urlcolor=cyan,
}

\renewcommand{\bottomtitlespace}{3cm}

\newcommand{\header}{
	ЛЕТЕН ТУРНИР ПО ИНФОРМАТИКА\\
	Русе, 7-9 юни 2024 г.\\
	Група A, 11 – 12 клас
}
\newcommand{\tl}{$0.15$ сек.}
\newcommand{\ml}{$256$ MB}

\begin{document}
	
\problem{Задача AK1. Скоби}

Иво решава задача с низ от $N$ правилно балансирани скоби, но не може да чете добре и се притеснява, че ще прочете някой подниз наобратно. Чуди се по колко начина може да се случи това, така че низът да остане правилно балансиран.

По-точно, даден е низ от $N$ символа, които са или \verb|(|, или \verb|)|. Низът е правилно балансиран, т.е. има еднакъв брой отварящи и затварящи скоби, на всяка отваряща съотества различна затваряща вдясно от нея и тези двойки отварящи-затварящи скоби не се пресичат, т.е. ако първата двойка е $A_1, B_1$, а втората $A_2, B_2$, такива че $A_1 < A_2$, е вярно, че или $A_1 < B_1 < A_2 < B_2$, или $A_1 < B_2 < A_2 < B_2$. 

Иво се чуди колко различни непразни подниза има, които могат да се обърнат в обратен ред (\textbf{без индвидуалните скоби да се обръщат}), така че целият низ все още да е правилно балансиран. Два подниза са различни, ако се различават позициите $L$ (на левите им краища) и/или $R$ (на десните). Целият низ също е подниз.

Например, ако низът е \verb|(())()|, възможните обръщания са 14:

\begin{itemize}
\item $L = R = 1$; получава се \verb|(())()|.
\item $L = R = 2$; получава се \verb|(())()|.
\item $L = R = 3$; получава се \verb|(())()|.
\item $L = R = 4$; получава се \verb|(())()|.
\item $L = R = 5$; получава се \verb|(())()|.
\item $L = R = 6$; получава се \verb|(())()|.
\item $L = 1$ и $R = 2$; получава се \verb|(())()|.
\item $L = 2$ и $R = 3$; получава се \verb|()()()|.
\item $L = 3$ и $R = 4$; получава се \verb|(())()|.
\item $L = 4$ и $R = 5$; получава се \verb|(()())|.
\item $L = 3$ и $R = 5$; получава се \verb|((()))|.
\item $L = 4$ и $R = 6$; получава се \verb|(())()|.
\item $L = 2$ и $R = 5$; получава се \verb|(())()|.
\item $L = 3$ и $R = 6$; получава се \verb|(()())|.
\end{itemize}

Напишете програма, която отговаря на запитването на Иво, т.е. тя трябва да смята въпросната бройка поднизове, които могат да бъдат обърнати без низът да стане небалансиран.

\subsection{Вход}
На първия ред на стандартния вход се въвежда едно число $N$ -- дължината на низа. На втория ред се въвежда низът без интервали. 

\subsection{Изход}
На единствен ред на стандартния изход изведете търсената бройка поднизове.

\subsection{Ограничения}
\vspace{0.1em}
\begin{itemize}
\item $2 \leq N \leq 4 \times 10^6$
\end{itemize}

\subsection{Подзадачи}

\begin{table}[H]
\begin{tblr}{|Q[c,m]|Q[c,m]|Q[c,m]|Q[c,m]|Q[c,m]|Q[c,m]|}
\hline
\textbf{Подзадача} & \textbf{Точки} & \textbf{$N \leq$} \\
\hline
$1$ & $7$ & $5 \times 10^2$ \\
\hline
$2$ & $9$ & $3 \times 10^3$ \\
\hline
$3$ & $11$ & $1.5 \times 10^4$ \\
\hline
$4$ & $30$ & $3 \times 10^5$ \\ 
\hline
$5$ & $21$ & $1.5 \times 10^6$ \\ 
\hline
$6$ & $22$ & $4 \times 10^6$ \\ 
\hline
\end{tblr}
\caption*{Точките за дадена подзадача се получават само ако се преминат успешно всички тестове в нея и всички предишни подзадачи.}
\end{table}
\vspace{0.5em}

\subsection{Пример}

\begin{table}[H]
\begin{tblr}{|m{3cm}|m{3cm}|}
\hline
\textbf{Вход} & \textbf{Изход} \\
\hline
\texttt{6\\
(())()} &
\texttt{14\\
} \\
\hline
\end{tblr}
\end{table}

\end{document}
