\documentclass[12pt]{article}
\newcommand{\bottomMargin}{2cm}

\newcommand{\header}{
	НАЦИОНАЛНА ОЛИМПИАДА ПО ИНФОРМАТИКА\\
	Национален кръг\\
	Хасково, 8-10 март 2024 г.\\
	Група AB, 9 – 12 клас, Ден 1
}

\newcommand{\tl}{$1.4$ сек.}
\newcommand{\ml}{$1024$ MB}

%---DOCUMENT MARGINS---
\usepackage{geometry} % Required for adjusting page dimensions and margins
\geometry{
	paper=a4paper, % Paper size, change to letterpaper for US letter size
	top=4cm, % Top margin
	bottom=\bottomMargin, % Bottom margin
	left=2cm, % Left margin
	right=2cm, % Right margin
	headheight=3cm, % Header height
	%footskip=1.5cm, % Space from the bottom margin to the baseline of the footer
	headsep=0.5cm, % Space from the top margin to the baseline of the header
	%showframe, % Uncomment to show how the type block is set on the page
}
\usepackage{cmap}

\usepackage[T2A]{fontenc}
\usepackage[bulgarian]{babel}
\usepackage{fontspec}
\setmainfont{Times New Roman}
\setsansfont{Times New Roman}
\setmonofont{Courier New}
\usepackage[math-style=TeX]{unicode-math}
\setmathfont{Latin Modern Math}

\usepackage[nobottomtitles*]{titlesec}
\titleformat
{\section} % command
{\normalfont\fontsize{14}{14}\sffamily\bfseries} % format
{} % label
{0pt} % sep
{} % before-code
\titlespacing{\section}{0pt}{0em}{0em}
\usepackage[dvipsnames]{xcolor}
\titleformat
{\subsection} % command
{\fontsize{14}{14}\itshape} % format
{} % label
{0pt} % sep
{} % before-code
[\vspace{-1em}{\color{LimeGreen}\rule{0.2\textwidth}{0.2em}}\vspace{-0.7em}] % after-code
\titlespacing{\subsection}{0pt}{0.5em}{0em}

\setlength{\parskip}{0.5em}
\setlength{\parindent}{24pt}
\sloppy

\usepackage{fancyhdr}
\pagestyle{fancy}
\usepackage{setspace}
\fancyhead[L]{
	\begin{minipage}{\textwidth}
		\includegraphics[width=2.8cm]{./logo.jpg}
	\end{minipage}
}
\fancyhead[C]{
	\begin{minipage}{\textwidth}
		\centering\large{\bf{\header}}
		\vspace{-0.35cm}
	\end{minipage}
}
\usepackage{emoji}
\usepackage{makecell}
\usepackage{tabularray}
\AtBeginEnvironment{table}{\vspace{-0.2cm}}
\AtEndEnvironment{table}{\vspace{-0.2cm}}
\usepackage{float}
\fancyhead[R]{
	\begin{tabular}{r@{\hspace{0.2cm}}l}
		\emoji{hourglass-not-done}: & \tl \\ 
		\emoji{floppy-disk}: & \ml \\ 
	\end{tabular}%
}
\renewcommand{\headrulewidth}{0cm}
\fancyheadoffset[L]{1cm}
\fancyheadoffset[R]{1cm}

\raggedbottom

\usepackage{amsmath}
\usepackage{stmaryrd}

\usepackage{graphicx}
\graphicspath{{./}}
\usepackage[export]{adjustbox}
\usepackage{wrapfig}
\makeatletter
\patchcmd\WF@putfigmaybe{\lower\intextsep}{}{}{\fail}
\AddToHook{env/wrapfigure/begin}{\setlength{\intextsep}{0pt}}
\makeatother
\usepackage[inkscapearea=page,inkscapepath=./svg-inkscape]{svg}
\svgpath{{./}}

\usepackage{placeins}
\usepackage{caption}
\captionsetup[table]{
	skip=0.25em,font=it,
	singlelinecheck=false,justification=justified,indention=-24pt,
	margin={24pt, 0pt}
}

\usepackage{enumitem}
\setlist{itemsep=-0.4em,leftmargin=\parindent,topsep=-\parskip}
\newcommand{\tabitem}{\indent~~\llap{\textbullet}~~}

\usepackage{hyperref}
\hypersetup{
	colorlinks=true,
	citecolor=blue,
	linkcolor=blue,
	urlcolor=cyan,
}


\begin{document}
	
\section{Задача AB3. GPUs}

Вие, като един модерен предприемач, сте основали generative AI startup. Процесът по генериране на изображения, текстове и т.н. е разбит на $N$ задачи, всяка от които отнема по една секунда на по едно GPU (графична карта). Предварително знаете кога ще стане налична всяка от задачите -- задача $i$ в секунда $T_i$. Имате достъп до външен суперкомпютър с $M$ на брой GPU-та, но всяко от тях си има различна цена за употреба на секунда -- GPU $j$ струва $C_j$ на секунда. Трябва да насрочите всяка от задачите $i$ за конкретно GPU $j$ в конкретна секунда, така че тази секунда да не е преди $T_i$ и да няма друга задача насрочена за същото време и GPU. Отново, \textbf{едно GPU работи по най-много една задача в дадена секунда}.

Нека финалното време на приключване (т.е. най късното насрочено време, плюс едно) е $F$, а общата платена сума е $S$, т.е. ако задача $i$ е насрочена за GPU $G_i$, $S = C_{G_1} + C_{G_2} + \dotsb + C_{G_N}$. Трябва да намерите минималната възможна стойност на $F \times S$. Ще трябва да решите $Q$ отделни, независими инстанции на задачата.

\subsection{Ограничения}
\vspace{0.1em}
\begin{itemize}
    \item $1 \leq N \leq 10^7$
    \item $1 \leq M \leq N$
    \item $0 \leq T_i \leq N$
    \item $1 \leq C_i \leq 2N$
    \item $1 \leq Q \leq 5$
\end{itemize}

\subsection{Интеракция}

Задачата е интерактивна. Вместо да четете и пишете от стандартните вход и изход, трябва само да напишете функция \verb|solveGpus| със следния тип:

\begin{verbatim}
__int128 solveGpus(
    std::vector<int>& gpuCosts,
    std::vector<int>& reqTimes);
\end{verbatim}

\textbf{Двата вектора от стойности, които функцията получава, ще бъдат сортирани в ненамаляващ ред}. Тя може да модифицира входните вектори. Функцията връща тип \verb|__int128|, който представлява 128-битово целочислено число. Това е нужно, защото отговорът може да надхвърля лимитите на \verb|long long|. Функцията може да се вика множество пъти. Всяко викане е независима инстанция на задачата.

Във Вашия код не трябва да има \verb|main| функция, но може да има всякакви други помощни функции, класове, променливи и т.н. Кодът Ви трябва да включва 
\verb|header| файла \verb|gpus.h|, в който, за Ваше удобство, също е дефиниран и оператор за извеждане на стойности от типа \verb|__int128|. Това става със следната директива към предпроцесора:

\begin{verbatim}
#include "gpus.h"
\end{verbatim}

Вашият код ще се компилира заедно с грейдър, който ще чете от входа и ще пише на изхода. На оценяващата система, единственото време, което се брои към времевия лимит, е времето прекарано в изпълнение на Вашия код, т.е. времето за вход и изход не се брои.

За локално тестване са Ви предоставени локален грейдър \verb|Lgrader.cpp| и копие на файла \verb|gpus.h|. Трябва да компилирате Вашия код заедно с локалния грейдър, за да го тествате. Това може да стане като ги сложите в една папка и използвате командата:

\begin{verbatim}
g++ -O2 -std=c++17 -Wl,--stack,1073741824 -Wall gpus.cpp Lgrader.cpp
-o gpus.exe
\end{verbatim}

\subsection{Подзадачи}
\begin{table}[H]
\begin{tblr}{|Q[c,m]|Q[c,m]|Q[c,m]|Q[c,m]|}
    \hline
    \textbf{Подзадача} & \textbf{Точки} & \textbf{$N \leq$} & \textbf{$Q \leq$} \\
    \hline
    $1$ & $10$ & $10$ & $1$ \\ 
    \hline
    $2$ & $8$ & $800$ & $2$ \\ 
    \hline
    $3$ & $13$ & $2200$ & $2$ \\ 
    \hline
    $4$ & $14$ & $10^4$ & $2$ \\ 
    \hline
    $5$ & $11$ & $10^5$ & $2$ \\
    \hline
    $6$ & $15$ & $10^6$ & $5$ \\
    \hline
    $7$ & $29$ & $10^7$ & $5$ \\
    \hline
\end{tblr}
\caption*{Точките за дадена подзадача се получават само ако се преминат успешно всички тестове, предвидени за нея.}
\end{table}
\FloatBarrier

\subsection{Пример}

Примерна следва входния формат на локалния грейдър ($Q$, а след това, за всеки тест: $N$, $M$, всички $C_j$, всички $T_i$).

\begin{table}[H]
\begin{tblr}{|X[30,l]|X[15,l]|X[55,j]|}
    \hline
    \textbf{Вход} & \textbf{Изход} & \textbf{Обяснение на примера} \\
    \hline
    \texttt{1 \\
    8 4 \\
    1 2 2 6 \\
    0 0 0 0 1 2 2 2 \\
    }
    & 
    \texttt{39}
    &
    {Първите 3 задачи в секунда 0, следващите 2 в секунда 1 и последните 3 в секунда 2. \\
    $S = 13$ \\
    $F = 3$} \\
    \hline
\end{tblr}
\end{table}
\FloatBarrier
	
\end{document}